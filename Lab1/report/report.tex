\include{settings}

\begin{document}	% начало документа

% Титульная страница
\include{titlepage}

% Содержание
\include{ToC}


\section{Цель работы}

Построить поведенческую модель программы по структурной модели. Программа должна принимать на вход файл с исходным кодом, написанном на языке программирования Java и выдавать в качестве результата файл с описанием поведенческой модели для всех методов, описанных в поданном файле. Вбранные модели следующие:
\begin{itemize}
\item \textbf{Структурная модель:} абстрактное синтаксическое дерево (AST)
\item \textbf{Поведенческая модель:} граф потока управления (CFG)
\end{itemize}

\section{Теоретическая информация}

\subsection{AST}

Абстрактное синтаксическое дерево --- помеченное ориентированное дерево, в котором внутренние вершины сопоставлены с операторами языка программирования, а листья --- с соответствующими операндами. Таким образом, листья являются пустыми операторами и представляют только переменные и константы. 

Синтаксические деревья используются в парсерах для промежуточного представления программы между деревом разбора и структурой данных, которая за этим используется в качестве внутреннего представления в компиляторе или интерпретаторе компьютерной программы для оптимизации и генерации кода. 

\subsection{CFG}

Граф потока управления --- множество всех возможных путей исполнения программы, представленное в виде графа. 

В графе потока управления каждый узел графа соответствует базовому блоку --- прямолинейному участку кода, не содержащему в себе ни операций передачи управления, ни точек, на которые управление передается из других частей программы. Имеется лишь два исключения:
\begin{itemize}
\item точка, на которую выполняется переход, является первой инструкцией в базовом блоке
\item базовый блок завершается инструкцией перехода
\end{itemize}

\section{Ход выполнения работы}

Общий план выполнения работы:
\begin{itemize}
\item Получение AST из исходного кода на Java
\item Построение CFG из AST
\item Экспорт CFG в PNG
\end{itemize}

\subsection{Взаимодействие с AST}

Для построения AST из исходного кода была использована часть из IDE Intellij Idea. Данная часть позволяет построить и модифицировать AST. Для того, чтобы обойти все интересующие узлы AST и получить узлы, соответствующие методам, необходимо реализовать рекурсивный метод. Пример приведен в листинге \ref{code:recursive}. 

\lstinputlisting[
	label=code:recursive,
	caption={Рекурсивный метод обхода},
]{recursive.java}
\parindent=1cm

\subsection{Построение CFG}

Для построения графа потока управления были созданы следующие элементы, которые указаны в листинге \ref{code:elems}.

\lstinputlisting[
	label=code:elems,
	caption={Классы и структуры, необходимые для построения CFG},
]{elems.java}
\parindent=1cm

Класс GraphElement - узел графа. В узле хранятся:

\begin{itemize}
\item ASTEntry - узел дерева, которому соответствует данный узел графа
\item ElementShape - значение перечисления, отвечающее за форму узла графа
\end{itemize}

Для построения CFG был реализован следующий класс (листинг \ref{code:cfgbuilder}).

\lstinputlisting[
	label=code:cfgbuilder,
	caption={Представление графа},
]{ControlFlowGraph.java}
\parindent=1cm

Данный класс строит CFG для каждого метода из AST и отображает граф. В данном классе, для построения CFG, происходит обход узлов, которые могут быть в теле метода и соответствующим образом обрабатывается каждый из них.

\subsection{Визуализация CFG}

Для визуализации полученных результатов используеся библиотека JavaFX, с помощью которой граф жкспортируется в .png формат. Класс, котороый реализует функции визуализации приведен в листинге \ref{code:visualization}.

\lstinputlisting[
	label=code:visualization,
	caption={Визуализация графа},
]{BuildFigure.java}
\parindent=1cm

\subsection{Результаты работы}

Проверка корректности работы программы проверялась на следующем тестовом входном файле (листинг \ref{code:test}).

\lstinputlisting[
	label=code:test,
	caption={Тестовый файл},
]{test.java}
\parindent=1cm

\newpage
Результаты построения CFG для метода из предыдущего листинга показаны на рисунке \ref{pic:test}
\begin{figure}[H]
	\begin{center}
		\includegraphics[scale=0.4]{test}
		\caption{Тестовый пример} 
		\label{pic:test} % название для ссылок внутри кода
	\end{center}
\end{figure}

\newpage
\subsection{Выводы}

\end{document}
