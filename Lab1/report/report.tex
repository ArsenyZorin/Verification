\include{settings}

\begin{document}	% начало документа

% Титульная страница
\include{titlepage}

% Содержание
\include{ToC}


\section{Цель работы}

Построить поведенческую модель программы по структурной модели. Программа должна принимать на вход файл с исходным кодом, написанном на языке программирования Java и выдавать в качестве результата файл с описанием поведенческой модели для всех методов, описанных в поданном файле. Вбранные модели следующие:
\begin{itemize}
\item \textbf{Структурная модель:} абстрактное синтаксическое дерево (AST)
\item \textbf{Поведенческая модель:} граф потока управления (CFG)
\end{itemize}

\section{Теоретическая информация}

\subsection{AST}

Абстрактное синтаксическое дерево --- помеченное ориентированное дерево, в котором внутренние вершины сопоставлены с операторами языка программирования, а листья --- с соответствующими операндами. Таким образом, листья являются пустыми операторами и представляют только переменные и константы. 

Синтаксические деревья используются в парсерах для промежуточного представления программы между деревом разбора и структурой данных, которая за этим используется в качестве внутреннего представления в компиляторе или интерпретаторе компьютерной программы для оптимизации и генерации кода. 

\subsection{CFG}

Граф потока управления --- множество всех возможных путей исполнения программы, представленное в виде графа. 

В графе потока управления каждый узел графа соответствует базовому блоку --- прямолинейному участку кода, не содержащему в себе ни операций передачи управления, ни точек, на которые управление передается из других частей программы. Имеется лишь два исключения:
\begin{itemize}
\item точка, на которую выполняется переход, является первой инструкцией в базовом блоке
\item базовый блок завершается инструкцией перехода
\end{itemize}

\section{Ход выполнения работы}

Общий план выполнения работы:
\begin{itemize}
\item Получение AST из исходного кода на Java
\item Построение CFG из AST
\item Экспорт CFG в PNG
\end{itemize}

\subsection{Взаимодействие с AST}

Для построения AST из исходного кода была использована часть из IDE Intellij Idea. Данная часть позволяет построить и модифицировать AST. Для того, чтобы обойти все интересующие узлы AST и получить узлы, соответствующие методам, необходимо реализовать рекурсивный метод. Пример приведен в листинге \ref{code:recursive}. 

\lstinputlisting[
	label=code:recursive,
	caption={Рекурсивный метод обхода},
]{recursive.java}
\parindent=1cm

\subsection{Построение CFG}

Для построения графа потока управления были созданы следующие элементы, которые указаны в листинге \ref{code:elems}.


\subsection{Визуализация CFG}
\subsection{Результаты работы}
\subsection{Выводы}

\begin{itemize}
\item первый элемент списка
\item второй элемент списка
\end{itemize}


\subsection{Картинка}

%\begin{figure}[H]
%	\begin{center}
%		\includegraphics[scale=0.7]{sample}
%		\caption{название картинки} 
%		\label{pic:pic_name} % название для ссылок внутри кода
%	\end{center}
%\end{figure}


\subsection{Листинг}

\lstinputlisting[
	label=code:hello,
	caption={hell\_o.c},% для печати символ '_' требует выходной символ '\'
]{hell_o.c}
\parindent=1cm % командна \lstinputlisting сбивает параментры отступа
Текст без отступа (следует за вставкой)

Новый параграф

\noindent Новый параграф с принудительно выключенным отступом


\subsection{Частичный листинг}
% настрока частичного ввода (требуется один раз)

\subsection{Таблица}

\begin{table}[H]
	\caption{ Название таблицы}
	\begin{center}
		\begin{tabular}{|l|l|}
			\hline
			top left & top right\\ \hline
			bot left & bot right\\ \hline
		\end{tabular}
		\label{tabular:tab_examp}
	\end{center}
\end{table}

\section{Выводы}

Исключения, пожалуй, составляют таблицы, так как их значительно сложнее создавать кодом, нежели в графическом редакторе. Но здесь никто не запрещает использовать визуальные средства создания таблиц для \LaTeX\ .
\end{document}
